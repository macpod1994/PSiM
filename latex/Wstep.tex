\section{Wstęp}
Celem projektu było stworzenie generatora sygnału wzorcowego na podstawie zadanej charakterystyki częstotliwościowej. Do realizacji zadania postanowiono skorzystać ze sprzętu laboratoryjnego, na który składały się:
\begin{enumerate}
\item Komputer klasy PC,
\item Zestaw startowy z mikrokontrolerem z rodziny STM32 (STM32F401),
\item Oscyloskop.
\end{enumerate}
Korzystano z oprogramowania znajdującego się na PC, na którym zainstalowany jest system operacyjny Windows, a w szczególności z:
\begin{enumerate}
\item STM32CubeMX - graficznego środowiska umożliwiającego szybką konfigurację układów peryferialnych wraz generację szablonu kodu w języku C,
\item SW4STM32 - środowiska umożliwiającego szybkie tworzenie oraz debugowanie napisanego kodu,
\item MATLAB - zintegrowanego środowiska programistycznego, w którym stworzono prototyp algorytmu oraz interfejs u użytkownika.
\end{enumerate}
Zdecydowano, że docelowo użytkownik zadawał będzie punkty charakterystyki z poziomu aplikacji napisanej w MATLABie. Informacje te przesłane zostaną do mikrokontrolera poprzez interfejs szeregowy zgodnie ze standardem RS-232. Mikrokontroler wygeneruje odpowiedni sygnał wzorcowy, a następnie wyśle go do przetwornika cyfrowo-analogowego. Sygnał wyjściowy możliwy będzie do obejrzenia na oscyloskopie.





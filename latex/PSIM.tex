\documentclass[11pt,a4paper,titlepage]{article}
\usepackage[utf8]{inputenc}
\usepackage[english,polish]{babel}
\usepackage[T1]{fontenc}
\usepackage{polski}
\usepackage[math,light]{anttor}
\usepackage{amsmath}
\usepackage{amsfonts}
%\usepackage{amssymb}
\usepackage{graphicx}
\usepackage{sidecap}
%\usepackage{wrapfig}
\usepackage{epstopdf}
\usepackage{booktabs}
\usepackage{forloop}
\usepackage[left=3cm,right=3cm,top=3cm,bottom=3cm]{geometry}
\usepackage[framed,numbered,autolinebreaks]{mcode}
\usepackage[colorlinks=false,hidelinks,urlcolor=blue,citecolor=green]{hyperref}
\usepackage{fancyhdr}
\usepackage{lastpage}
\usepackage{array}
\usepackage{hhline}
\usepackage{multirow}
\usepackage {float}
\usepackage{enumerate}%[I], numerki, [(a)]
%ustawienie poziomów wypunktowania do wyboru: $\bullet$, $\cdot$, $\diamond$, $-$, $\ast$ and $\circ$ 
\renewcommand{\labelitemi}{$\diamond$}
\renewcommand{\labelitemii}{$\bullet$}
\renewcommand{\labelitemiii}{$-$}
\renewcommand{\labelitemiv}{$\ast$}


\AtBeginDocument{

	\renewcommand{\tablename}{Tabela}

	\renewcommand{\figurename}{Rys.}
}

%tabelki
\usepackage{tabularx}
\newcolumntype{A}{>{\centering\arraybackslash}X}
\newcolumntype{B}{>{\centering\arraybackslash} m{0.4\textwidth} }




\usepackage{csquotes}
\DeclareQuoteAlias{croatian}{polish} % Ponieważ `csquotes` nie posiada polskiego stylu, można skorzystać z mocno zbliżonego stylu chorwackiego.

%\addbibresource{bibliografia.bib}

\pagestyle{fancy}
\fancyhf{}
\fancyhead[R]{\slshape{\small \rightmark}}
\fancyfoot[R]{Wahadło odwrócone}
\fancyhead[L]{M. Cebula, P. Merynda, M. Podsiadło}     
\fancyfoot[L]{Strona \thepage \hspace{1pt} z\hspace{1pt} \pageref*{LastPage}}    
\renewcommand{\headrulewidth}{1pt}
\renewcommand{\footrulewidth}{1pt}


\begin{document}

\begin{titlepage}

\newcommand{\HRule}{\rule{\linewidth}{0.5mm}} % Defines a new command for the horizontal lines, change thickness here

\center % Center everything on the page
 
%----------------------------------------------------------------------------------------
%	HEADING SECTIONS
%----------------------------------------------------------------------------------------

\textsc{\LARGE Akademia Górniczo - Hutnicza im. Stanisława Staszica}\\[0.5cm]
\includegraphics[scale=0.6]{agh}\\[1cm] % Name of your university/college
\textsc{\Large Wydział Elektrotechniki, Automatyki, Informatyki i Inżynierii Biomedycznej}\\[0.5cm] % Major heading such as course name
\textsc{\large Katedra Automatyki i Inżynierii Biomedycznej}\\[0.5cm]
\textsc{ Kierunek: Automatyka i robotyka}\\[0.5cm] % Minor heading such as course title

%----------------------------------------------------------------------------------------
%	TITLE SECTION
%----------------------------------------------------------------------------------------

\HRule \\[0.4cm]
{ \huge \bfseries Procesory Sygnałowe i Mikrokontrolery\\[1cm]Generator sygnału wzorcowego na podstawie zadanej charakterystyki częstotliwościowej}\\[0.4cm] % Title of your document
\HRule \\[1.0cm]
 


%----------------------------------------------------------------------------------------
%	DATE SECTION
%----------------------------------------------------------------------------------------

%{\large \today}\\[1.5cm] % Date, change the \today to a set date if you want to be precise

%----------------------------------------------------------------------------------------
%	LOGO SECTION
%----------------------------------------------------------------------------------------

%\includegraphics[height=70mm]{agh.jpg}%\\[1cm] % Include a department/university logo - this will require the graphicx package
%----------------------------------------------------------------------------------------
%	AUTHOR SECTION
%----------------------------------------------------------------------------------------

\begin{flushleft}
\Large
\emph{Wykonali:}\\
Maciej Cebula\\
Piotr Merynda\\
Maciej Podsiadło\\[1cm]

% If you don't want a supervisor, uncomment the two lines below and remove the section above
 \emph{Prowadzący:}\\
dr inż. Tomasz Dziwiński\\[3cm] % Your name
 
\end{flushleft}
%----------------------------------------------------------------------------------------
\end{titlepage}
\clearpage
\setcounter{page}{2}

\section{Wstęp}
Celem projektu było stworzenie generatora sygnału wzorcowego na podstawie zadanej charakterystyki częstotliwościowej. Do realizacji zadania postanowiono skorzystać ze sprzętu laboratoryjnego, na który składały się:
\begin{enumerate}
\item Komputer klasy PC,
\item Zestaw startowy z mikrokontrolerem z rodziny STM32 (STM32F401),
\item Oscyloskop.
\end{enumerate}
Korzystano z oprogramowania znajdującego się na PC, na którym zainstalowany jest system operacyjny Windows, a w szczególności z:
\begin{enumerate}
\item STM32CubeMX - graficznego środowiska umożliwiającego szybką konfigurację układów peryferialnych wraz generację szablonu kodu w języku C,
\item SW4STM32 - środowiska umożliwiającego szybkie tworzenie oraz debugowanie napisanego kodu,
\item MATLAB - zintegrowanego środowiska programistycznego, w którym stworzono prototyp algorytmu oraz interfejs u użytkownika.
\end{enumerate}
Zdecydowano, że docelowo użytkownik zadawał będzie punkty charakterystyki z poziomu aplikacji napisanej w MATLABie. Informacje te przesłane zostaną do mikrokontrolera poprzez interfejs szeregowy zgodnie ze standardem RS-232. Mikrokontroler wygeneruje odpowiedni sygnał wzorcowy, a następnie wyśle go do przetwornika cyfrowo-analogowego. Sygnał wyjściowy możliwy będzie do obejrzenia na oscyloskopie.





\section{Konfiguracja}
Konfiguracja wszystkich peryferiów wymaganych do uruchomienia układu została przeprowadzona w programie \textit{CubeMX}. Następnie na jej podstawie wygenerowano projekt do dalszego rozwijania w środowisku \textit{Eclipse}. Na rysunku \ref{stmPorty} przedstawiono konfiguracje poszczególnych portów mikrokontrolera.

\begin{figure}[h!]
	\centering
	\includegraphics[scale = 0.4]{fig/stmPorty.png}
	\caption		
	{Konfiguracja portów mikrokontrolera Stm32 w programie \textit{CubeMX}.}
	\label{stmPorty}
\end{figure}


\subsection{Konfiguracja timerów}
Do cyklicznego generowania kolejnych próbek sygnału wyjściowego wykorzystano przerwanie od Timera nr 11. Licznik skonfigurowano tak aby generował przerwanie z częstotliwością 100 kHz. Na rysunku \ref{timer11} pokazany jest panel konfiguracyjny timer-a z programu \textit{CubeMX}, natomiast na rysunku \ref{stmclock} konfiguracja sygnałów zegarowych w całym układzie.  

\begin{figure}[h!]
	\centering
	\includegraphics[scale = 0.8]{fig/timer11.png}
	\caption		
	{Przykład konfiguracji timera 11.}
	\label{timer11}
\end{figure}

\begin{figure}[h!]
	\centering
	\includegraphics[scale = 0.5]{fig/clock.png}
	\caption		
	{Konfiguracja sygnałów zegarowych w układzie.}
	\label{stmclock}
\end{figure}

\subsection{Konfiguracja przetwornika DAC}
Przetwornik DAC został skonfigurowany w trybie 12 bitowym, a do generacji sygnału wyjściowego wykorzystywany był pierwszy kanał przetwornika na pinie PA4. Z racji na ograniczoną rozdzielczość przetwornika zdecydowano się na ograniczenie wartości amplitudy generowanego sygnału do $\pm 50$. Ponieważ wartości poszczególnych punktów postaci czasowej sygnału mogą przyjmować wartości zarówno dodatnie jak i ujemne, to przyjęto, że wartość zero odpowiada liczbie 2048 wpisanej do rejestru przetwornika.   
\subsection{Konfiguracja portu szeregowego}

Do przesyłania danych z komputera do mikrokontrolera wykorzystany został protokół szeregowy \textit{RS232}. Parametry charakteryzujące transmisję podane są w tabeli \ref{tab_rs232}.

\begin{table}[h]
	\caption{Parametry portu szeregowego.}
	\label{tab_rs232}
	\centering
	
	\begin{tabular}{|c|c|}
		\hline
		\textbf{Parametr} & \textbf{Wartość}\\
		\hline
		Prędkość transmisji & 115200 Bitów/s \\
		\hline
		Długość słowa & 8 bitów \\
		\hline
		Bit parzystości & - \\
		\hline
		Bit stopu & 1 \\
		\hline
	\end{tabular}
\end{table}   
\section{Algorytm}
\label{algorytm}
\subsection{Dyskretna transformacja Fouriera}
Głównym narzędziem matematycznym, które wykorzystano do rozwiązania postawionego zadania była odwrotna transformacja Fouriera. Przesłana przez użytkownika charakterystyka częstotliwościowa reprezentowana będzie poprzez wektor próbek. Dodatkowo biorąc pod uwagę fizyczne ograniczenia, w tym niezerowy czas działania przetwornika postanowiono wykorzystać algorytm odwrotnej dyskretnej transformacji Fouriera (IDFT). Zakłada się, że przesyłana charakterystyka odpowiada sygnałowi rzeczywistemu oraz największa częstotliwość w niej występująca jest mniejsza od połowy częstotliwości próbkowania algorytmu. 

Oznaczając prze $N$ liczbę punktów charakterystyki częstotliwościowej zadanych przez użytkownika wprowadzono następujące oznaczenia:
\begin{align*} 
&freq = [f_1, f_2, ..., f_N]^T \\ 
&amp = [A_1, A_2, ..., A_N]^T \\
&phase = [\phi_1, \phi_2, ..., \phi_N]^T
\end{align*}
gdzie $freq$ jest wektorem niezerowych częstotliwości składowych zadanego sygnału, natomiast $amp$ oraz $phase$ odpowiadających im amplitud i przesunięć fazowych. Uwzględniając założenia:
\begin{equation}
\label{eq:twP}
f_p > \frac{max(freq)}{2}
\end{equation}
można w sposób zwarty zapisać formułę:
\begin{equation}
\label{eq:alg}
x(k\Delta t) = \sum_{n=1}^{N}A_n\cdot\cos(2 \pi f_nk \Delta t - \phi_n), \quad k=0,1,2,...
\end{equation}
umożliwiającej wyliczenie kolejnych wartości sygnału spróbkowanego z częstotliwością $f_p=\frac{1}{\Delta t}$, którego charakterystyka częstotliwościowa została zadana.

\subsection{Prototyp algorytmu}
Przed przystąpieniem do implementacji algorytmu \ref{eq:alg} na mikrokontrolerze stworzono jego prototyp w środowisku MATLAB. Z uwagi na postać \ref{eq:alg} programowa realizacja algorytmu przebiegła sprawnie. Rezultaty przedstawione zostały w podrozdziale \ref{GUI}.


\section{Implementacja}
\subsection{Interfejs użytkownika}
\label{GUI}
Zgodnie z założeniami projektu zbudowano aplikację w środowisku MATLAB umożliwiającą użytkownikowi szybkie przesyłanie danych do mikrokontrolera. Zadbano o intuicyjność graficznego interfejsu użytkownika udostępniając następujące funkcjonalności:
\begin{enumerate}
	
	\item Ręczne wprowadzanie kolejnych punktów charakterystyki.
	\item Automatyczne wprowadzanie częstotliwości po kliknięciu użytkownika na wykres.
	\item Wykres amplitudy oraz przesunięcia fazowego dla wprowadzonych punktów charakterystyki częstotliwościowej.
	\item Algorytm liczący sygnał w dziedzinie czasu na podstawie wprowadzonych przez użytkownika punktów (zgodnie z równaniem \ref{eq:alg}).
	\item Rysowanie odpowiedzi w dziedzinie czasu \textit{online}.
	\item Automatyczne skalowanie wykresów (skala liniowa, logarytmiczna).
	\item Możliwość wysłania wprowadzonej charakterystyki do mikrokontrolera (zgodnie z zaimplementowanym protokołem opisanym w podrozdziale \ref{PK}).
\end{enumerate} 
Po wprowadzeniu dwóch próbek aplikacja prezentuje się następująco:
\begin{figure}[h!]
	\centering
	\includegraphics[scale = 0.3]{fig/GUI2.png}
	\caption		
	{Widok aplikacji po wprowadzeniu dwóch punktów na charakterystyce częstotliwościowej.}
	\label{gui1}
\end{figure}
Zakres wyświetlanych częstotliwości zgodny jest z założeniem opisanym równaniem \ref{eq:twP}. Po kliknięciu w zakładkę \textit{GUI} w menu użytkownik może wybrać jedną z opcji skalowania wykresów, a także zresetować aktualnie wprowadzone punkty lub zakończyć działanie aplikacji w sposób zapewniający zwolnienie wszystkich jej zasobów.
\begin{figure}[h!]
	\centering
	\includegraphics[scale = 0.3]{fig/ls.png}
	\caption		
	{Wyświetlenie wprowadzonej charakterystyki w skali logarytmicznej.}
	\label{log}
\end{figure}
\begin{figure}[h!]
	\centering
	\includegraphics[scale = 0.3]{fig/as.png}
	\caption		
	{Wyświetlenie charakterystyki w skali liniowej - automatycznie dopasowanej do wprowadzonych punktów.}
	\label{as}
\end{figure}
W każdej chwili możliwe jest wygenerowanie wykresu w nowym oknie w sposób umożliwiający łatwy zapis, a także porównanie z sygnałem wyliczonym prze mikrokontroler. Po naciśnięciu przycisku \textit{Plot time domain signal} pojawi się wykres \ref{tds}.
\begin{figure}[h!]
	\centering
	\includegraphics[scale = 0.3]{fig/tds.png}
	\caption		
	{Wykres w dziedzinie czasu.}
	\label{tds}
\end{figure}
Użytkownik po wciśnięciu przycisku \textit{Put} może ręcznie wprowadzić kolejny punkt charakterystyki poprzez najechanie kursorem w odpowiednie miejsce na wykresach - amplitudy oraz przesunięcia fazowego.
\begin{figure}[h!]
	\centering
	\includegraphics[scale = 0.3]{fig/put.png}
	\caption		
	{Wprowadzanie punktów do charakterystyki częstotliwościowej za pomocą kursora.}
	\label{fig:put}
\end{figure}
Po wciśnięciu lewego przycisku myszy w sytuacji zobrazowanej na rysunku \ref{fig:put} główny ekran aplikacji prezentuje się następująco - rys. \ref{fig:after}.
\begin{figure}[h!]
	\centering
	\includegraphics[scale = 0.3]{fig/after.png}
	\caption		
	{Ekran aplikacji po wprowadzeniu punktu za pomocą kursora.}
	\label{fig:after}
\end{figure}
\subsection{Protokół komunikacyjny}
\label{PK}
Dane z komputera do mikrokontrolera przesyłane są za pomocą protokołu \textit{RS232}, którego parametry podano w tabeli \ref{tab_rs232}. Specyfika implementowanego algorytmu wymagała zaproponowania jednolitego sposobu przesyłania danych opisujących zadaną charakterystykę. \\
Należało uwzględnić następujące czynniki:
\begin{enumerate}
	
	\item możliwość przesyłania dowolnie wielu zestawów danych opisujących pojedynczy punk charakterystyki, tzn. częstotliwość, amplitudę i przesunięcie fazowe,
	\item możliwość przesyłania nowego zestawu danych bez konieczności resetowania układu.
	\item liczy opisujące poszczególne współczynniki mogą mieć różną liczbę cyfr.
\end{enumerate}
%
%
Biorąc pod uwagę wszystkie powyższe wymagania oraz specyfikę protokołu \textit{RS232} zdecydowano się przesyłać dane w postaci pojedynczych znaków. Procedura \textit{składania} liczby z przesłanych znaków została zaimplementowana w samym urządzeniu. Aby algorytm mógł rozróżnić czy przesłany znak jest częścią obecnie składanej liczby czy dotyczy już następnego parametru, wprowadzono znak \textit{tabulacji}, który oddziela cyfry kolejnych liczb.
\\
Przyjęto, że pierwsza liczba przesłana do mikrokontrolera określa liczbę punktów charakterystyki. Następnie przesyłane są kolejne zestawy danych składające się z trzech liczb: częstotliwości, amplitudy i przesunięcia fazowego. Każda z tych trzech liczb zapisywana jest w odpowiedniej tablicy.
\\
Do przechowywania poszczególnych wartości wykorzystywane są dwa zestawy dynamicznie alokowanych tablic. Pierwszy zestaw służy do przechowywania danych obecnie generowanego sygnału, natomiast drugi wykorzystywany jest do zapisywania aktualnie przesyłanego zestawu. 
\\
Po przesłaniu wszystkich liczb pamięć przeznaczona na pierwszy zestaw tablic jest zwalniana. Do wska\'zników, które wcześniej przechowywały adresy pierwszego zestawu przepisywane są adresy zestawu drugiego. Dzięki operacjom na wska\'znikach zaoszczędzono czas potrzebny na przepisywanie danych z jednego zestawu do drugiego. 
\\
Kod od przerwania odebrania nowego znaku zamieszczony jest na listingu \ref{uart_rec}
\begin{lstlisting}[frame=single, caption = Implementacja protokołu komunikacji, label = uart_rec]
void HAL_UART_RxCpltCallback(UART_HandleTypeDef *huart) {

static int i=0;
static int sample_iter = -1;
static int liczbaProbek = -1;
static int idTab = 0;
uint8_t data[50];
uint16_t size = 0;

// procedura "skladania" liczby z przeslanych cyfr

if(Received == 9 || i==ReceivedTabSize) // zatwierdzenie liczby TABem
{
volatile uint8_t recNumber = 0;
for (int k=0;k<i;k++)
{
recNumber += power(i-k-1) * ReceivedTab[k];
}
i = 0;

//odebranie pierwzej liczby okreslajacej liczbe przesylanych 
//punktow
if(sample_iter == -1)
{
liczbaProbek = recNumber;
MAX = liczbaProbek;
freqTabRT = malloc(liczbaProbek*sizeof(uint8_t));
ampTabRT = malloc(liczbaProbek*sizeof(uint8_t));
phaseTabRT = malloc(liczbaProbek*sizeof(uint8_t));
sample_iter = 0;
}
else
{
size = sprintf(data, "echoProbki:\t %u \n\r", recNumber);
HAL_UART_Transmit_IT(&huart2, data, size);
if(sample_iter%3 == 0) // freq
{
freqTabRT[idTab] =  recNumber;
freq = recNumber;
}
else if(sample_iter%3 == 1) // amp
{
ampTabRT[idTab] = recNumber;
}
else if(sample_iter%3 == 2) // phase
{
phaseTabRT[idTab] = recNumber;
idTab++;
}
sample_iter += 1;
}
if(idTab == liczbaProbek)
{
free(freqTab);
free(ampTab);
free(phaseTab);
freqTab = freqTabRT;
ampTab = ampTabRT;
phaseTab = phaseTabRT;

size = sprintf(data, "koniec \n\r");
HAL_UART_Transmit_IT(&huart2, data, size);
sample_iter = -1;
idTab = 0;
}

}
else
{
ReceivedTab[i] = atoi(&Received);
i++;
}

// ponowne wlaczenie odbierania
HAL_UART_Receive_IT(&huart2, &Received, r_size);
}

int power(int a)
{
int wynik = 1;
for (int i=0;i<a;i++)
wynik *= 10;
return wynik;
}
\end{lstlisting}
\subsection{Generacja sygnału wyjściowego}

Sygnał wyjściowy generowany jest na podstawie algorytmu opisanego w rozdziale \ref{algorytm}, cyklicznie w przerwaniu od timera nr 11. Kod wyliczający odwrotną transformatę przedstawiony jest listingu \ref{fourierKod}. 
%
\begin{lstlisting}[frame=single, caption = Kod funkcji obliczającej odwrotną transformatę Fouriera w przerwaniu od timera 11., label = fourierKod]
void HAL_TIM_PeriodElapsedCallback(TIM_HandleTypeDef *htim)
{
	static long long int t = 0;
	int index = 0;
	
	float v;
	
	if(htim->Instance == TIM11){
	
		HAL_DAC_SetValue(&hdac,DAC_CHANNEL_1,DAC_ALIGN_12B_R,output);
		if(freqTab==NULL || ampTab==NULL ||phaseTab==NULL)
		{
			return;
		}
		out = 0;
		t=t+1;
		for (int ii=0;ii<MAX;ii++)
		{
			v = 360 * tp * freqTab[ii]*t + 180 - phaseTab[ii];
			cos = cosApproximation(v);
			out = out + ampTab[ii]*cos;
		}
		
		// skalowanie amplitudy -50 do 50 na od 0 do 4095
		output = (int)(out*40.96 + 2048);
	}
}
\end{lstlisting}
%
Aby zaoszczędzić czas potrzebny na wyliczanie wartości cosinusa kąta w każdej iteracji pętli algorytmu zdecydowano się na zapisanie wartości tej funkcji w pamięci mikokontrolera z rozdzielczością $1^0$. Rozdzielczość ta jest w wielu wypadkach zadawalająca, jednak w przypadkach wymagających większej precyzji można skorzystać z funkcji \mbox{\textit{cosApproximation}}, której kod podano w listingu \ref{cosApp}, a która to aproksymuje wartość cosinusa kąta funkcją liniową pomiędzy dwiema stablicowanymi wartościami. \\

\begin{lstlisting}[frame=single, caption = Funkcja aproksymująca wartość cosinusa., label = cosApp]
inline float cosApproximation(float v)
{
	float value = 0;
	
	uint16_t a = ((int) v) % 360;
	uint16_t b = (a + 1) % 360;
	
	float cos_a = cosTable[a];
	float cos_b = cosTable[b];
	wsp_a = cos_b - cos_a;
	wsp_b = cos_a - wsp_a * a;
	
	value = wsp_a * (v - (int)(v) + a) + wsp_b;
	
	return value;
}
\end{lstlisting}

\section{Wnioski}
\indent W celu analizy porównawczej sygnałów uzyskanych na wyjściu z przetwornika DAC oraz sygnałów wzorcowych wygenerowanych w środowisku Matlab zapisano przesłane na oscyloskopie przebiegi czasowe w formacie danych *.CSV. Następnie importując je do środowiska Matlab możliwe było przeskalowanie wcześniej wygenerowanych przebiegów wzorcowych. Po przeskalowaniu i dopasowaniu fazy możliwe było zestawienie na jednym wykresie syganłów zadanych oraz sygnałów wygenerowanych przez układ. Wyniki kilku eksperymentów przedstawione są na rysunkach \ref{fig:10_100}, \ref{fig:10_100_1000} i \ref{fig:50_60}, gdzie kolorem niebieskim zaznaczony jest przebieg wygenerowany przez układ, natomiast kolorem czerwonym przebieg wzorcowy. Parametry podane jako wejście do układu zestawione zostały w tabelach \ref{tab_10_100}, \ref{tab_10_100_1000} i \ref{tab_50_60}.

\begin{table}[h]
	\caption{Parametry pierwszego przebiegu.}
	\label{tab_10_100}
	\centering
	
	\begin{tabular}{|c|c|c|}
		\hline
		\textbf{Częstotliwość [Hz]} & \textbf{Amplituda} & \textbf{Przesunięcie fazowe [\textdegree]}\\
		\hline
		10 & 7 & 180 \\
		\hline
		100 & 2 & 0 \\
		\hline
	\end{tabular}
\end{table}

\begin{figure}[h!]
	\centering
	\includegraphics[scale = 0.8]{fig/10_100.png}
	\caption		
	{Zestawienie przebiegu wygenerowanego przez układ z przebiegiem wzorcowym dla parametrów z tabeli \ref{tab_10_100}.}
	\label{fig:10_100}
\end{figure}

\begin{table}[h]
	\caption{Parametry drugiego przebiegu.}
	\label{tab_10_100_1000}
	\centering
	
	\begin{tabular}{|c|c|c|}
		\hline
		\textbf{Częstotliwość [Hz]} & \textbf{Amplituda} & \textbf{Przesunięcie fazowe [\textdegree]}\\
		\hline
		10 & 7 & 180 \\
		\hline
		100 & 2 & 0 \\
		\hline
		1000 & 2 & 0 \\
		\hline
	\end{tabular}
\end{table}

\begin{figure}[h!]
	\centering
	\includegraphics[scale = 0.8]{fig/10_100_1000.png}
	\caption		
	{Zestawienie przebiegu wygenerowanego przez układ z przebiegiem wzorcowym dla parametrów z tabeli \ref{tab_10_100_1000}}
	\label{fig:10_100_1000}
\end{figure}

\begin{table}[h]
	\caption{Parametry trzeciego przebiegu.}
	\label{tab_50_60}
	\centering
	
	\begin{tabular}{|c|c|c|}
		\hline
		\textbf{Częstotliwość [Hz]} & \textbf{Amplituda} & \textbf{Przesunięcie fazowe [\textdegree]}\\
		\hline
		50 & 4 & 180 \\
		\hline
		60 & 3 & 90 \\
		\hline
	\end{tabular}
\end{table}

\begin{figure}[H]
	\centering
	\includegraphics[scale = 0.8]{fig/50_60.png}
	\caption		
	{Zestawienie przebiegu wygenerowanego przez układ z przebiegiem wzorcowym dla parametrów z tabeli \ref{tab_50_60}}
	\label{fig:50_60}
\end{figure}

\indent Analizując przedstaione wyniki można zauważyć stosunkowo dobre dopasowanie wygenerowanych sygnałów do sygnałów wzorcowych. Widoczne na wykresie \ref{fig:50_60} ostre zmiany wartości wyjściowej tłumaczyć można jako niedokładność generowaną na zbliżeniu przez oscyloskop. Rozdzielczość, w której zapisane zostały serie danych wynosi 600x600 co generuje widoczne niedokładności.

\indent W ogólnym rozrachunku udało się zrealizowac wszystkie założenia projektowe. Realizację algorytmu genaracji sygnału na podstawie danych parametrów. Wysyłanie wyliczonych wartości na układ DAC. Interfejs użytkownika umożliwiający zadawanie parametrów sygnału oraz komunikację z mikrokontrolerem.

\newpage
\tableofcontents
\newpage
\nocite{*}
\end{document}
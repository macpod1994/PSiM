\section{Algorytm}
\label{algorytm}
\subsection{Dyskretna transformacja Fouriera}
Głównym narzędziem matematycznym, które wykorzystano do rozwiązania postawionego zadania była odwrotna transformacja Fouriera. Przesłana przez użytkownika charakterystyka częstotliwościowa reprezentowana będzie poprzez wektor próbek. Dodatkowo biorąc pod uwagę fizyczne ograniczenia, w tym niezerowy czas działania przetwornika postanowiono wykorzystać algorytm odwrotnej dyskretnej transformacji Fouriera (IDFT). Zakłada się, że przesyłana charakterystyka odpowiada sygnałowi rzeczywistemu oraz największa częstotliwość w niej występująca jest mniejsza od połowy częstotliwości próbkowania algorytmu. 

Oznaczając prze $N$ liczbę punktów charakterystyki częstotliwościowej zadanych przez użytkownika wprowadzono następujące oznaczenia:
\begin{align*} 
&freq = [f_1, f_2, ..., f_N]^T \\ 
&amp = [A_1, A_2, ..., A_N]^T \\
&phase = [\phi_1, \phi_2, ..., \phi_N]^T
\end{align*}
gdzie $freq$ jest wektorem niezerowych częstotliwości składowych zadanego sygnału, natomiast $amp$ oraz $phase$ odpowiadających im amplitud i przesunięć fazowych. Uwzględniając założenia:
\begin{equation}
\label{eq:twP}
f_p > \frac{max(freq)}{2}
\end{equation}
można w sposób zwarty zapisać formułę:
\begin{equation}
\label{eq:alg}
x(k\Delta t) = \sum_{n=1}^{N}A_n\cdot\cos(2 \pi f_nk \Delta t - \phi_n), \quad k=0,1,2,...
\end{equation}
umożliwiającej wyliczenie kolejnych wartości sygnału spróbkowanego z częstotliwością $f_p=\frac{1}{\Delta t}$, którego charakterystyka częstotliwościowa została zadana.

\subsection{Prototyp algorytmu}
Przed przystąpieniem do implementacji algorytmu \ref{eq:alg} na mikrokontrolerze stworzono jego prototyp w środowisku MATLAB. Z uwagi na postać \ref{eq:alg} programowa realizacja algorytmu przebiegła sprawnie. Rezultaty przedstawione zostały w podrozdziale \ref{GUI}.

